\documentclass{article}
\usepackage[utf8]{inputenc}

\title{document.tex}
\author{a.arone }
\date{October 2018}

\begin{document}

\maketitle

\section{Introduction}

Logiciels libres : une introduction

\section{Qu'est ce qu'un logiciel ?}

Pour comprendre le concept de logiciel libre, vous devez d'abord comprendre ce qu'est un logiciel (ou encore programme ou application).

Du point de vue de l'utilisateur, un logiciel est une application qui répond à l'un de ses besoins (traitement de textes, programme de dessin, jeu, ...). C'est une suite de petites instructions invisibles pour l'utilisateur, qui forment un tout cohérent.

Ces logiciels ont besoin d'un système d'exploitation pour fonctionner. Le système d'exploitation permet d'accéder aux ressources de la machine (lecteur de disquette, écran, clavier, ...). C'est aussi le système d'exploitation qui se charge d'exécuter le logiciel.

Les instructions d'un logiciel sont écrites dans un langage que l'ordinateur peut comprendre, le langage machine (ou langage binaire). En revanche, il est très difficile (voir impossible) à un humain de comprendre ce langage (c'est de plus interdit dans de nombreux pays). La manière la plus courante de créer un logiciel est de l'écrire dans un langage informatique compréhensible par des humains, et ensuite de le traduire vers le langage binaire. Cette traduction est effectuée par un logiciel appelé compilateur.

Le logiciel dans sa forme compréhensible est appelé code source, et dans sa version en langage machine binaire.
\section{Qu'est-ce qu'un logiciel libre ?}

Imaginez que vous vous trouviez dans un restaurant et que vous mangiez un excellent plat. Peut-être aurez vous l'envie de le cuisiner chez vous pour vos amis ?

C'est impossible, car vous n'avez pas la recette du plat. Vous pouvez toujours le manger dans le restaurant, mais même si vous connaissez le goût, vous ne savez comment le reproduire.

En informatique, c'est la même chose avec un logiciel. Le code source est la recette, le binaire est le plat cuisiné. La plupart des logiciels sont distribués sans leur code source, et il est interdit d'essayer de comprendre leur fonctionnement. Il est interdit de les partager avec vos amis, et il est interdit d'essayer de les modifier pour les adapter à vos besoins.

Un logiciel libre vous garantit plusieurs libertés :
\begin{itemize}
    \item la liberté de copier le logiciel pour vous ou vos amis
    \item la liberté de comprendre son fonctionnement si vous le désirez
    \item la liberté de le modifier et de distribuer vos modifications
\end{itemize}
\section{La FSF et le projet GNU}

Richard Stallman, considéré par tous comme le père des logiciels libres a commencé à penser au logiciel libre lorsqu'il travaillait au laboratoire d'intelligence artificielle au MIT. Le laboratoire possédait une imprimante qui tombait souvent en panne, mais comme les chercheurs avaient le code source du pilote\note{Le pilote est le logiciel qui sert à accéder et à utiliser l'imprimante.} de l'imprimante, ils avaient modifié le programme pour que l'imprimante leur envoie un signal à chaque panne.

Un beau jour, le laboratoire achete une nouvelle imprimante de marque Xérox, plus fiable. En revanche, le pilote de l'imprimante n'est pas fourni à la livraison. Richard Stallman entend parler plus tard d'un laboratoire qui posséde les sources de ce pilote. S'y rendant, on lui répond que le laboratoire s'était engagé à ne pas diffuser les sources du pilote. Ressentant ce refus comme une agression, Richard Stallman prend alors conscience du danger de la logique propriétaire.

Richard Stallman décide alors de fonder la Free Software Foundation. Conscient qu'il est impossible d'utiliser un ordinateur sans utiliser de logiciels propriétaires, il initialise le premier projet de la fondation, le projet GNU (GNU est un jeu de mots récursif signifiant GNU's Not Unix). Ce projet vise à concevoir un système d'exploitation complet et entièrement libre. Ce système sera compatible avec UNIX, mais sera différent. Aujourd'hui ce système existe, et s'appelle GNU/Linux.

Pour valider ce système, une base légale est nécessaire. Cette base légale est la GNU GPL, pour GNU General Public Licence. La GNU GPL est la licence des logiciels libres par excellence. Elle détermine des conditions de distribution qui garantissent les libertés de l'utilisateur. Un programme protégé par la GPL est libre, mais la GPL impose aussi que tout travail dérivé de ce logiciel reste libre.
IV  GNU/Linux, une réalité

Le projet GNU a commencé de zéro, et ses premiers développements avaient pour but de créer un environnement de développement utilisable. Les premiers efforts se sont donc très logiquement portés sur l'élaboration d'un compilateur (gcc), d'un éditeur de textes (Emacs), ainsi que d'un déboggeur (gdb). Un noyau (Hurd) était aussi prévu, mais son élaboration prend plus de temps que prévu.

En 1991, Linus Torvalds commence à écrire son noyau, qui connaîtra plus tard le succès que l'on sait. Le projet GNU l'adopte en attendant que Hurd soit opérationnel, et le noyau Linux devient alors composant du système GNU/Linux.

Le système GNU/Linux est réputé pour sa fiabilité et sa robustesse. Ceci est dû en partie à la liberté des logiciels qui le composent, l'accès aux sources permettant de corriger très facilement et rapidement une erreur de programmation.

Le système GNU/Linux est fortement POSIX (POSIX est un ensemble de normes définissant un système UNIX idéal), ce qui le rend très similaire à la plupart des UNIX propriétaires existants. Le noyau Linux est multi-tâches, multi-utilisateurs, et intègre la plupart des technologies les plus récentes (SMP, clustering, RAID, ...).

GNU/Linux est aujourd'hui utilisé aussi bien par les entreprises que par les utilisateurs finaux.

De nombreuses applications sont disponibles pour l'utilisateur non informaticien. Par exemple, l'environnement graphique Gnome permet d'utiliser son ordinateur sans utiliser la ligne de commande. Gimp est un logiciel de traitement d'images très puissant. Gnumeric est un tableur intégré au projet Gnome. Ce ne sont que des exemples, de nombreuses applications peuvent rendre d'immenses services à l'utilisateur. La sécurité et la fiabilité de ce système sont un argument supplémentaire pour l'utiliser (les virus n'existent pas). Plusieurs distributions existent, et l'installation du système GNU/Linux est désormais facile.

Pour l'entreprise, les champs d'application sont vastes. La robustesse du système couplée à l'excellence des applications orientées réseau (Apache par exemple) en fait un système de choix pour les serveurs (le couple GNU/Linux et Apache est le plus utilisé au monde en matière de serveurs web). Les outils de développement (en particuliers les outils GNU : Emacs, gcc, cvs, ...) font de GNU/Linux une plate-forme de développement puissante et agréable à utiliser. En dehors de ces considérations techniques, l'utilisation de logiciels libres affranchit de plus l'entreprise de la main-mise d'un éditeur.
\section{Pourquoi le logiciel libre est t'il meilleur ?}
\subsection{Pour le technicien}

Par définition, les sources d'un logiciel libre sont publiques et librement utilisables. Pour le technicien, c'est d'un intérêt inconsidérable. D'abord, il peut se permettre d'utiliser un code source codé par un autre programmeur sans crainte de poursuites. Il n'a pas besoin de ``réinventer la roue''.

De plus, les erreurs d'un logiciel libre seront plus rapidement corrigées. L'ouverture des sources permet aux informaticiens (ou tout simplement aux utilisateurs ayant quelques connaissances) de corriger eux-même les erreurs qu'ils rencontrent dans les programmes qu'ils utilisent. Les corrections seront intégrées dans la version suivante du logiciel.

Pour les étudiants en informatique ou même les professionnels, l'accès aux sources permet un apprentissage des plus profonds. Pour beaucoup de professionnels, ``la meilleure documentation qui puisse exister, ce sont les sources elles-mêmes''.
\subection{V.b Pour l'utilisateur}

L'utilisateur ne bénéficie par directement de l'accès aux sources. La plupart des automobilistes ne connaissent pas le fonctionnement d'un moteur, c'est la même chose en informatique. Maintenant, imaginons un monde ou 10% de la population fait de la mécanique pendant ses moments perdus, et passe son temps à améliorer son véhicule. Si vous achetez une voiture d'occasion, vous êtes sûr que le moteur a été amélioré par son propriétaire précédent. On gagne en fiabilité, en performance et en qualité.

En informatique, on a plus ou moins le même concept, mais là vous avez une voiture neuve qui profite de toutes les technologies développées dans des petits garages indépendants par des artisants habiles, et dont les meilleures sont rassemblées dans votre véhicule. Vous êtes sûr de la qualité tout comme de la sécurité. Dans le cas d'un vice de conception, vous êtes certain que quelqu'un aura détecté l'erreur avant vous et l'aura corrigée.
\subsection{Pour les entreprises}

Comme pour les utilisateurs, les entreprises bénéficient de la qualité du logiciel. Mais en plus, le modèle de développement des logiciels libres est un avantage à lui tout seul. Il permet de corriger très rapidement une erreur de programmation dont les effets sont critiques pour l'entreprise, et aucune logique marchande ne conditionne les sorties des différentes versions des logiciels libres.

Si une entreprise décide de jouer le jeu et de développer des logiciels libres, elle bénéficie du support de toute la communauté du logiciel libre, tout comme elle garantit la pérénité de son programme.
\section{Brevets sur les logiciels}

D'après la directive européenne du 14 mai 1991 concernant la protection juridique des programmes d'ordinateurs, ceux-ci sont protégés par le droit d'auteur, en tant qu'oeuvre littéraire au sens de la convention de Berne pour la protection des oeuvres littéraires et artistiques.

La directive précise que seule l'expression d'un programme d'ordinateur est protégée et que les idées et les principes qui sont à la base de la logique, des algorithmes et des langages de programmation ne sont pas protégés par le droit d'auteur. De même, les idées et principes qui sont à la base de ses interfaces ne sont pas protégés par le droit d'auteur.

L'unique critère pour déterminer si un programme peut bénéficier d'une protection est son originalité, en ce sens que ce programme est la création intellectuelle propre à son auteur.

Un projet de directive européenne propose de remettre en cause la non brevetabilité des programmes d'ordinateurs garantie par la directive européenne de 1991 et la Convention de Munich.
\subsection{Effets dangereux des brevets sur les logiciels}

Quelques effets pervers des brevets sur les logiciels :
\subsubsection{Pour l'innovation}

Le brevet permet de restreindre l'utilisation d'une innovation pendant vingt ans.

Or, dans le cas des programmes d'ordinateurs, les brevets résultent le plus souvent de la découverte ou de la simple application d'une propriété mathématique ou d'un algorithme, qui fait partie du patrimoine de l'humanité. La plupart du temps, les découvertes effectuées par un programmeur ne sont que l'adaptation d'idées déjà existantes à un problème bien défini.

Aux Etats-Unis, plus de 50 000 brevets concernant l'informatique sont déposés. Dans un tel contexte, il est impossible à un programmeur d'accomplir son métier sans transgresser un brevet à son insu. Les brevets informatiques couvrent des domaines divers : les éléments les plus simples de l'interface d'un programme, des algorithmes qu'un étudiant de première année dans une école informatique pourrait inventer à nouveau, et même l'utilisation des couleurs. Il faudrait mettre un ou deux avocats derrière chaque programmeur, et, dans la plupart des cas, il serait impossible d'écrire un programme fonctionnel, car les algorithmes les plus efficaces ou évidents sont protégés par des brevets.

De grands groupes détiennent la majorité des brevets et s'en servent pour bloquer leurs concurrents, le brevet étant ici une arme offensive et non plus défensive. Seules les très grandes compagnies de logiciels auraient le droit à la création. Ce serait la fin du droit de chacun d'innover et de créer.
\subsubsection{Pour l'informatique libre}

L'informatique libre est forcément la plus touchée par la brevetabilité des programmes d'ordinateurs. Un logiciel libre utilise par nature des techniques et des inventions découvertes par autrui, tout comme il autorise l'utilisation future des techniques développées par son programmeur.

Un programmeur de logiciel libre ne peut pas non plus se permettre d'embaucher des avocats pour se protéger dans le cas probable où il enfreindrait un brevet par inadvertance.

De plus, l'informatique libre posant comme principe le partage et l'utilisation mutuelle des connaissances, pour le bien de la communauté, ne peut accepter la brevetabilité des programmes d'ordinateurs.

Les idées appartiennent à l'humanité.
\section{Pour en savoir plus}

Si vous voulez en savoir plus, n'hésitez pas à faire un tour sur les sites suivants :
\begin{itemize}
     \item http://www.gnu.org, le site web du projet GNU et de la FSF. C'est le site web de référence lorsqu'on parle de logiciel libre. Il contient la philosophie GNU qui explique pourquoi il faut du logiciel libre et de nombreuses autres choses comme la GPL.
    \item     http://www.april.org, le site de l'association APRIL (Association pour la Promotion et la Recherche en Informatique Libre). Le web contient des informations sur l'association, sur leurs travaux en cours, sur les manifestations organisées par l'association ainsi que sur la philosophie de l'informatique libre.
    \item    http://www.debian.org, le site de la distribution de GNU/Linux la plus libre.
    \item     http://www.linux.org, le site de départ pour toute information ou recherche de logiciels concernant le système GNU/Linux.
    \item     http://www.linux-france.org, site d'informations (avec de nombreuses documentations) en français autour de GNU/Linux.
    \item    http://www.linux-center.fr, site d'actualités en français autour de GNU/Linux.
    \item     http://lpf.ai.mit.edu, le site de la League For Programing Freedom, qui s'intéresse au problème des brevets sur les logiciels.
\end{itemize}


Une formule :

\begin{equation}
   f(x) = \sum_{k=1}^n \frac{1}{k^2} \label{intelligent}
\end{equation}

\begin{table}[h!]
    \centering
    \begin{tabular}{c|c}
       5  & 6 \\
      4   & 3
    \end{tabular}
    \caption{Quelques valeurs}
    \label{tab:my_label}
\end{table}

\end{document}